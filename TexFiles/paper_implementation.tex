
\section{Implementation}
%Hash
\subsection{Hash-Based Commitments}
For an easy introduction into the implementations, let's first introduce the hash-based commitments. 
These implementations rely solely on the cryptographic attributes of their regarding hash-function, making them relatively easy to understand, as every cryptocraphic process is \textit{veiled} by a single function - leaving out complex mathematics.

It's therefore important, to choose a \textbf{cryptographic hash function} ( see \cite{DeHa15} section 2.2.1 ). A \textit{normal} hash-function does not fulfill the attributes as shown further below, and therefore cannot be used for commitments. 

~\newline The basic protocol implemented with hash-functions:
	\begin{enumerate}
		\item Alice chooses a random value $s$
		\item Alice produces $h = Hash(m \star s)$ and sends $h$ and $Hash$ to Bob
		\item Bob keeps $<Alice,h,Hash>$
		\item Alice reveals herself by sending Bob $m$ and $s$
		\item Bob checks if $Hash(m \star s) \equiv h$
	\end{enumerate}
~\newline This implementation is parallel to the example with the box: The \textit{key} to the box is the random salt, and the message is hidden behind it. 

To choose a random salt value is necessary, as the domain of the messages is usually limited. Picking up the example of the coin-toss, Alice would be only able to commit to \textit{Tail} or \textit{Head}. Without a random value, Bob (and any Eve) could simply try both values and compare to the commitment. 

While the example of coin-tossing is rather trivial, even bigger example such as \textit{dates of birth} can be easily tried for multiple centuries. 

Additionally, without the salt, using rainbow-tables and other dictionaries is possible. 

~\newline It's also to mention that the message $m$ should never be \textit{really} valuable - as it's send in cleartext in \textit{step 4} of the implementation.  

~\newline The attributes required for a correct commitment-scheme are inherited directly by the attributes of a cryptographic hash-function (see \cite{Mironov05} section 4.1 paragraph 3f): 
~\newline \textbf{Binding:} After Alice created the hash, due to the \textbf{second-pre-image resistance} (see \cite{DeHa15} section 2.2.1 proposition 2.11 ) of the hash-function, she won't find any second message in feasible time that produces the same hash \footnote{Bob needs to verify the used hashfunction. There are known hash-collisions to some hash-functions, which can be used to intentionally produce failures}. Therefore, she is bound to her value, as any other message would produce a different hash.  
~\newline \textbf{Hiding:} After Bob recieved the hash, due to the \textbf{pre-image resistance} (see \cite{DeHa15} section 2.2.1 proposition 2.12 )of the hash-function, the only way Bob can get to know $m$ is by trying every possible value. As mentioned above, it's necessary to add a random salt for this attribute to be guaranteed.

~\newline To end this implementation, let's summarize the benefits: 
\begin{itemize}
	\item the implementation is easy to understand without further knowledge of mathematics
	\item $iff$ the hash-function is cryptographic, the commitments are \textit{safe}
	\item it's possible to commit \textit{words} as messages, unlike other hash-functions, enabling \textit{human-readable} examples  
\end{itemize} 

%Pedersen
\subsection{Pedersen Commitments}
Instead of using hash-functions for their functionality, the pedersen-commitments (See \cite{Ped01} section 3 ) gain their security from the un-feasability to extract roots from a finite body build by two (large) prime numbers (see \cite{DeHa15} Algorithm A.40 paragraph 2). This assumption is also known as the decision diffie-hellman problem and is used for other common cryptographic protocols, such as the ElGamal-encryption (see \cite{b1} section 4.1 ). 

Even tough \textit{finding} these roots is as good as guessing (see \cite{b1} section 3.4 corollary 3.5) if not every required element is known, if the prime numbers are known verifying the result is trivial. 

Producing cypher-texts firsthand is also an easy task for Alice. 

~\newline Unlike the hashfunction, there is a bigger setup Bob needs to do before the main-protocol begins: 
	\begin{enumerate}
		\item choosing a large prime number $p$
		\item choosing a smaller prime number \newline $q \in \{1..p| q\div (p-1) = 0\}$
		\item choosing $g,v \in G_q \neq 1$
		\item sending Alice $p,q,g,v$ 
	\end{enumerate}
~\newline With these steps, Alice and Bob are sure to operate in the same finite body, which can be checked for \textit{computational safeness}. The produces $g$ is often called the $generator$, the $v$ often $valitador$.

Choosing the body can also be done by Alice, which would drastically benefit her, as she can purposely choose generators and validators which would enable her to construct non-unique commitments. 

As a simplified example, Alice could choose the same $g$ and $v$, making it possible for her to switch message and salt at will, while successfully revealing her commitment. 

Even if Bob notices $g=v$, and rejects these kind of tricks, Alice is able to produce variables for certain collisions, which Bob can only notice with brute-forcing. 

Therefore it's common for Bob to choose the body, as he is not about to commit values. If Alice notices problems with the safety of the given numbers, she can reject the communication.  

~\newline The implementation of the pedersen-commitments: 
	\begin{enumerate}
		\item Alice requests $p,q,g,v$ from Bob. \newline Alice checks that:
		\begin{itemize}
			\item $q,p$ are primes, 
			\item q divides p-1, 
			\item that $g,v \in G_q$. 
		\end{itemize}
		\item Alice chooses her message $m \in \{1..p\}$ and a random number $r \in \{1..q-1\}$
		\item Alice sends $c = g^rv^m$ to Bob \textbf{(commit)}
		\item Bob keeps $<Alice,c,<p,q,g,v>>$
		\item Alice can reveal herself by sending $r,m$ to Bob. ~\newline Bob checks $c = g^rv^m$
	\end{enumerate}

~\newline The \textbf{Hiding} is fulfilled with the discrete-log assumption, as noone who recieves the tupel $<p,q,g,v>$ can find the $r,m$ which produce $c$, without guessing or computing every possibility. Given large prime numbers, this is computational impossible. 
~\newline The \textbf{Binding}-attribute is only partly fulfilled: As shown above, if Alice would have the power to choose $<p,q,g,v>$ she would be able to produce a lot of collisions. It's still possible for her to find collisions on purpose, if Bob has chosen the finite body. 

However, if Bob has chosen large prime-numbers and distinct $g,v$, he can be relatively safe, as Alice will have troubles finding these collisions in reasonable time. This implementation can still be considered \textit{pro-Alice}, as she is put in a more powerful position.    

~\newline The primary benefits of the pedersen-commitments are the following: 
\begin{itemize}
	\item the commitments always contain random parts
	\item the computational security can easily be increased by choosing bigger prime-numbers
\end{itemize}
%QR
\subsection{Quadratic-Residues}
Another implementation of commitment-schemes is based on quadratic residues in an finite body, first introduced by Manuel Blum in 1982 (See \cite{Blum82}) and therefore (by far) the oldest commitment-scheme. In this scheme, Alice commits herself to a single bit (for reference, see \cite{DeHa15} section 4.3.1). 

~\newline The basic idea is to produce a finite body of size $n = p \cdot q $ where $p,q \in Primes$. A number $v$ of this body is quadratic if and only if it's quadratic in both $p$ and $q$. To commit to $true$, Alice sends $n$ and $v$, but\textbf{ not $p$ and $q$}, and chooses a quadratic $v$. If she wants to commit to false, she sends a non-quadratic $v$. 

~\newline While it's easy to compute whether a given number is quadratic in a prime using the \textit{legendre symbol} $(\dfrac{v}{p})$ (see \cite{PrGloLeg}), it's not possible to simply compute whether a number is quadratic in the product of two primes, except if both primes are known. 

~\newline Given any number $v \in \mathcal{Z}_n$, one could guess that is not a quadratic residue - this holds true for 75\% of all cases. 

As this makes \textit{guessing} the commitment better than simply guessing $true$ or $false$, we need to implement an additional security measure: The value $v$ must be chosen from the numbers in $n$, which have a positive Jakobi-Symbol (see \cite{PrGloJak}). 

Despite additional mathematic attributes, the Jakobi-Symbol can be simplified as the product of legendre-symbols, defining $(\dfrac{v}{n})= (\dfrac{v}{p})\cdot (\dfrac{v}{q})$.

There are two cases, in which the Jakobi-Symbol is positive: The value $v$ is quadratic in \textbf{both} $q$ and $p$, or the value $v$ is \textbf{not} quadratic in \textbf{both} $q$ and $p$. 

Thanks to this addition, \textit{guessing} whether a given value $v$ where $v \in J_n^{+1}$ is quadratic in $n$ has a chance of 50\% and is therefore not better than guessing $true$ or $false$. 

~\newline The only way to successfully calculate instead of guessing is to find one of the prime numbers by trying, which is not feasible given large prime numbers.

~\newline Based on these mathematical procedures, the protocol works as follows:
\begin{enumerate}
	\item Alice chooses primes $p,q$ and an element $v \in \{ x \in J_n^{+1}\}$ 
	\item Alice commits to a bit b by choosing a random number $r$ and sending Bob: $n,v$ and $c = r^2\cdot v^b$
	\item Bob verifies that $(\dfrac{v}{n})=1$ and keeps $<Alice,n,v,c>$
	\item Alice reveals herself by sending Bob $p,q,r,b$
	\item Bob verifies that $p,q$ are primes, $n=pq$ and $c=r^2\cdot v^b$ 
\end{enumerate}
This protocol enhances the above assumptions by hiding the real value and it's quadracy behind a random value $r$. Therefore the verification is a little more complex. 
~\newline
If A wants to commit a quadratic residue, she chooses a quadratic $v$ and $b=1$, therefore $r^2 \cdot v^1$ will be quadratic ~\newline
If a wants to commit a non-quadratic value, she chooses a non-quadratic $v$ and $b=0$, therefore $r^2 \cdot v^0 = r^2$ which is quadratic ~\newline
If $v$'s actual quadracy and $b$ do not match, $c$ won't be quadratic and therefore rejected by Bob. 
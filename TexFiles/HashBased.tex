\section{Hash-Based}
\begin{frame}
	\frametitle{Hash-Based Commitments}
	\framesubtitle{General Concept}
	\begin{LARGE}
		\begin{enumerate}
			\item Alice produces $h = Hash(m)$ and sends Bob $h$ and $Hash$
			\item Bob keeps $<Alice,h,Hash>$
			\item Alice reveals herself by sending Bob $m$
			\item Bob checks if $Hash(m) \equiv h$
		\end{enumerate}
	\end{LARGE}
	 ~\newline  ~\newline 
	 \begin{center}
	 	\textbf{Important: NEVER} use actual important data as message,  ~\newline  you send it as cleartext in Step 3. 
	 \end{center}
\end{frame}

\begin{frame}
	\frametitle{Hash-Based Commitments}
	\framesubtitle{Fullfillment of Attributes}
	\begin{LARGE}
	\textbf{Hiding:} because of the Hash-functions \textbf{Pre-Image resistance}, it's nearly impossible to find the message m from the hash. This holds true for any Bob and any Eve. ~\newline ~\newline
	\textbf{Binding:} because of the Hash-functions \textbf{collision-resistance}, it's nearly impossible to find another message m with the same hash.
	\end{LARGE}
\end{frame}

\begin{frame}
	\frametitle{Hash-Based Commitments}
	\framesubtitle{Problem: unlimited range - limited domain}
	Usually: Bob (and Eve) are not able to \textit{guess} $m$ from $h$ and $Hash$
	~\newline ~\newline
	But: if the \textit{plausible domain} of $m$ is known, its possible for modern computers to brute force reveal your $m$ 
	
	~\newline ~\newline 
	Example: Alice commits to Bob about the result of a soccer game Germany vs. Brazil. ~\newline Therefore she chooses a score of 0:7 and sends Bob $h = SHA_3(str(0:7))$ and the Hashfunction $SHA_3$ ~\newline Eve catches the commitment and knows the context of the soccer game. ~\newline she can know try reasonable combinations of results from 0:0 up to 20:20. She only needs to try $20 \cdot 20 = 400$ results 
\end{frame}

\begin{frame}
	\frametitle{Hash-Based Commitments}
	\framesubtitle{Salting the Hash}
	Improved Concept: 
	\begin{LARGE}
		\begin{enumerate}
			\item Alice chooses a random value $s$
			\item Alice produces $h = Hash(m,s)$ and sends $h$ and $Hash$ to Bob
			\item Bob keeps $<Alice,h,Hash>$
			\item Alice reveals herself by sending bob $m$ and $s$
			\item Bob checks if $Hash(m,s) \equiv h$
		\end{enumerate}
	\end{LARGE}
\end{frame}

\begin{frame}
	\frametitle{Hash-Based Commitments}
	\framesubtitle{Addition}
	\textbf{Alice is anonymus}. She never stated her name, used certificates, etc. \newline
	Alice can produce as many commitments for as many personas as she wants.  
	~\newline ~\newline
	For increased security: 
	\begin{itemize}
		\item commitments should be one-use only
		\item commitments should have a lifetime (in time and/or tries)
		\item traded commitments to a third party should be deprecated directly with first reveal
		\item messages must contain random parts
	\end{itemize}
\end{frame}
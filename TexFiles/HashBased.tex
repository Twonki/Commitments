\section{Hash-Based}
\begin{frame}
	\frametitle{Hash-Based Commitments}
	\framesubtitle{General Concept}
	
	\begin{enumerate}
		\item Alice produces $h = Hash(m)$ and sends Bob $h$ and $Hash$
		\item Bob keeps $h$ and $Hash$
		\item Alice reveals herself by sending Bob $m$
		\item Bob checks if $Hash(m) \equiv h$
	\end{enumerate}
	
\end{frame}

\begin{frame}
	\frametitle{Hash-Based Commitments}
	\framesubtitle{Problem: unlimited range - limited domain}
	Usually: Bob (and Eve) are not able to \textit{guess} $m$ from $h$ and $Hash$
	~\newline ~\newline
	But: if the \textit{plausible domain} of $m$ is known, its possible for modern computers to brute force reveal your $m$ 
	
	~\newline ~\newline 
	Example: Alice commits to Bob about the result of a soccer game Germany vs. Brazil. ~\newline Therefore she chooses a score of 0:7 and sends Bob $h = SHA_3(str(0:7))$ and the Hashfunction $SHA_3$ ~\newline Eve catches the commitment and knows the context of the soccer game. ~\newline she can know try reasonable combinations of results from 0:0 up to 20:20. She only needs to try $20 \cdot 20 = 400$ results 
\end{frame}

\begin{frame}
	\frametitle{Hash-Based Commitments}
	\framesubtitle{Salting the Hash}
	Improved Concept: 
	\begin{itemize}
		\item Alice chooses a random value $s$
		\item Alice produces $h = Hash(m,s)$ and sends $h$ and $Hash$ to Bob
		\item Bob keeps $h$ and $Hash$
		\item Alice reveals herself by sending bob $m$ and $s$
		\item Bob checks if $Hash(m,s) \equiv h$
	\end{itemize}
\end{frame}
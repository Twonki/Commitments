\section{Binary}

\begin{frame}
	\frametitle{Quadratic Residues}
	\framesubtitle{Definitions and Setup}
	Commitment to only {0,1} using quadratic residues. ~\newline
	
	A number n is \textbf{always} quadratic, if it's the product of two quadrats. ~\newline
	\begin{center}
		$p^2 * q ^2 = p * p * q * q = (p*q)^2$ 
	\end{center}

	choose primes $p,q$ and check for every element $x \in \mathrm{Z_n^*} , n=p*q$ ~\newline ~\newline
	
	\textbf{The only way to check if x is quadratic in n, is to check if its quadratic for p and q!}
\end{frame}

\begin{frame}
	\frametitle{Quadratic Residues}
	\framesubtitle{Jakobi-Matrizes}
	Required: Legendre-Symbol $(\dfrac{x}{p}) = x^{p-1} mod(p)$ ~\newline ~\newline
%	\begin{center}
%		$(\dfrac{x}{n} ) = (\dfrac{x}{p}) \cdot (\dfrac{x}{q})$ ~\newline ~\newline
%	\end{center}
	\begin{center}
			\begin{tabular}{|c|c|c|c|}
			\hline 
			$(\dfrac{x}{p})$&$(\dfrac{x}{q})$ &$(\dfrac{x}{n})$& quadratic \\ 
			\hline 
			1 & 1  & 1  & yes \\ 
			\hline 
			1 & -1  & -1  & no \\ 
			\hline 	
			-1 & 1  & -1  & no \\ 
			\hline 	
			-1 & -1  & 1  & no \\ 
			\hline 
		\end{tabular} 
	\end{center}
~\newline ~\newline
	given X and guessing we have 75\% cases where it's not quadratic. 
	~\newline ~\newline
	given X and $(\dfrac{x}{n})=1$ we have 50:50 Quadratic:nonQuadratic 
\end{frame}

\begin{frame}
	\frametitle{Quadratic Residues}
	\framesubtitle{Protocoll}
	\begin{LARGE}
		\begin{enumerate}
			\item Alice chooses primes $p,q$ and an element $v \in \{ x \in \mathbf{Z}_n | (\dfrac{x}{n}) = 1\}$ 
			\item Alice commits to a bit b by choosing a random numer $r$ and sending Bob: $n,v$ and $c = r^2\cdot v^b$
			\item Bob verifies that $(\dfrac{v}{n})=1$ and keeps it
			\item Alice reveals herself by sending Bob $p,q,r,b$
			\item Bob verifies that $p,q$ are primes, $n=pq$ and $c=r^2\cdot v^b$ 
		\end{enumerate}
	\end{LARGE}
~\newline
If A wants to commit a quadratic residue, she chooses a quadratic $v$ and $b=1$, therefore $r^2 \cdot v^1$ will be quadratic ~\newline
If a wants to commit a non-quadratic value, she chooses a nonquadratic $v$ and $b=0$, therefore $r^2 \cdot v^0 = r^2$ which is quadratic
\end{frame}
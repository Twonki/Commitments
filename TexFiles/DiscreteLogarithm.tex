\section{Discrete Log}
\begin{frame}
	\frametitle{Discrete Logarithm - Pedersen commitment scheme}
	\framesubtitle{Requirements and Definitions}
	Prerequisites: Bob needs to setup the environtment for alice, by 
	\begin{Large}
		\begin{enumerate}
			\item choosing a large prime number p
			\item choosing a smaller prime number $q \in \{1..p| q\div (p-1) = 0\}$
			\item choosing $g,v \in G_q \neq 1$
			\item sending Alice $p,q,g,v$ 
		\end{enumerate}
	\end{Large}
	~\newline
	Now Alice can \textit{build} the exact same group and subgroup like Bob. ~\newline
	This is similiar to sending the hash-function.  
\end{frame}

\begin{frame}
	\frametitle{Discrete Logarithm - Pedersen commitment scheme}
	\framesubtitle{Implementation}
	\begin{Large}
			\begin{enumerate}
			\item Alice requests $p,q,g,v$ from Bob. \newline Alice checks that:
			\begin{itemize}
				\item $q,p$ are primes, 
				\item q divides p-1, 
				\item that $g,v \in G_q$. 
			\end{itemize}
			\item Alice chooses her message $m \in \{1..p\}$ and a random number $r \in \{1..q-1\}$
			\item Alice sends $c = g^rv^m$ to Bob \textbf{(commit)}
			\item Bob keeps $<Alice,c,<p,q,g,v>>$
			\item Alice can reveal herself by sending $r,m$ to Bob. ~\newline Bob checks $c = g^rv^m$
		\end{enumerate}
	\end{Large}
\end{frame}

\begin{frame}
	\frametitle{Discrete Logarithm - Pedersen commitment scheme}
	\framesubtitle{Benefits}
	Major:
	\begin{itemize}
		\item Commitments always contain random parts
		\item No collision possible (unlike Hashfunctions)	
	\end{itemize}
	Minor:
	\begin{itemize}
		\item tupels are (usually) smaller to store than hashes
		\item $p,q,g,v$ are easily changed/renewed (you could not renew hashfunctions)
	\end{itemize}
\end{frame}
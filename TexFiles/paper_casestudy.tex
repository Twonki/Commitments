
\section{Case-Study: Migrating user privileges in web-applications}
\label{sec:casestudy}
To finalize this paper, a use-case for commitments is presented. 

This example describes how a one-time authentication for a web-application can be realized with commitments. 

There are common examples and closed-source libraries which enable developers to implement these features easily, however to keep this paper neutral, only the concept will be shown without going further into detail of several implementations. 

~\newline User-Privileges in web-applications are commonly granted by setting regarding attributes in the session-cookie. As the session-cookie is in the hands of a possibly malevolent user, one of the main-security-measures is to certify the cookie by the server. TODO: Quelle Nachrichten zertifizieren. This behaviour is similiar to signing messages in common message exchange. 

As the host usually verifies the cookie in every transaction, he is safe for two main attacks: The user can not create his own cookie, and he cannot alter given cookies. The only way to recieve a certified session-cookie is through a successful login.

~\newline Creating a safe login seems like an easy task - but is in itself one of the most important and maintenance-intensive parts of every web-page. Hardening the page, keeping everything up to date and enforce principles such as \textit{Deep Security} (TODO: Source!) requires not only knowledge but time. Therefore simplifying login-procedures, especial for distributed web-applications is a primary goal of web-development. 

~\newline For this cause commitment-schemes can be used to implement a simple \textit{challenge and response} protocol for one time authentication and privilege-migration. For the general setup, see the figure \ref{fig:usecaseone}: 

Alice is the user, Bob-1 is the server which has a functional (and safe) login. Bob-2 is known to Bob-1 and they trust each other. Alice final goal is to be logged in and use Bob-2's services. 
~\newline
~\newline  
The start of this protocol is shown in figure \ref{fig:usecaseone}: In the first step, Alice needs to be logged in and trusted by Bob-1. She then announces her wish to migrate to Bob-2 (e.g. by clicking on a hyperlink) and places a commitment. While these commitments vary from implementation to implementation, they usually contain a mixture of random variables, as well as common connection-attributes such as IP-adress and system-times.   
\begin{figure} [h]
	\centering
	\includegraphics[width=0.6\linewidth]{Images/UseCaseOne}
	\caption[Protocol Start]{Start: Alice log's in, announces her wish to migrate and deposits a commitment}
	\label{fig:usecaseone}
\end{figure}
After the commitment is received by Bob-1, he can start the migration as in figure \ref{fig:usecasetwo}. The first thing to migrate is the commitment itself - with a reference to Alice. This transmission does not need to be encrypted.  

Additionally a secure encrypted message about Alice privileges needs to be exchanged between Bobs. This can either be: 
\begin{enumerate}
	\item The session-cookie itself 
	\item a database reference
	\item a reference to Bob-2's internals (such as a simple role)
\end{enumerate}
Where most common is the first option, which also enables the possibility for a double-certification, if Bob-2 does not need additional attributes: He therefore signs the full, signed cookie from Bob-1 when granting it again, creating an certification-stack. 

\begin{figure}[h]
	\centering
	\includegraphics[width=0.6\linewidth]{Images/UseCaseTwo}
	\caption[Protocol Start]{Bob-1 migrates the commitment to Bob-2, as well as additional information for the privileges}
	\label{fig:usecasetwo}
\end{figure}
After Bob-2 successfully received the two messages, he can challenge Alice to reveal the commitment, as shown in figure \ref{fig:usecasethree}. If she is able to, she is granted to privileges regarding the second message. 

She is now able to interact with Bob-2 as if he would have had an usual login.

Most of this protocol can be simply done by user-scripts running in background - it's not required for Alice to enter any random numbers manually. These variables can be migrated with the browsercache. The Challenge can also be requested and resolved fully automaticly, creating the user-experience of a connected web-page.  
\begin{figure}[h]
	\centering
	\includegraphics[width=0.6\linewidth]{Images/UseCaseThree}
	\caption[Protocol Start]{Bob-2 challenges Alice to reveal herself, Alice reveals, Bob-2 grants a certified cookie}
	\label{fig:usecasethree}
\end{figure}

~\newline If the communication between parties is secured, and the commitment-scheme has a safe implementation, this one-time authentication is as-safe as the original login. 